\PassOptionsToPackage{unicode=true}{hyperref} % options for packages loaded elsewhere
\PassOptionsToPackage{hyphens}{url}
%
\documentclass[]{article}
\usepackage{lmodern}
\usepackage{amssymb,amsmath}
\usepackage{ifxetex,ifluatex}
\usepackage{fixltx2e} % provides \textsubscript
\ifnum 0\ifxetex 1\fi\ifluatex 1\fi=0 % if pdftex
  \usepackage[T1]{fontenc}
  \usepackage[utf8]{inputenc}
  \usepackage{textcomp} % provides euro and other symbols
\else % if luatex or xelatex
  \usepackage{unicode-math}
  \defaultfontfeatures{Ligatures=TeX,Scale=MatchLowercase}
\fi
% use upquote if available, for straight quotes in verbatim environments
\IfFileExists{upquote.sty}{\usepackage{upquote}}{}
% use microtype if available
\IfFileExists{microtype.sty}{%
\usepackage[]{microtype}
\UseMicrotypeSet[protrusion]{basicmath} % disable protrusion for tt fonts
}{}
\IfFileExists{parskip.sty}{%
\usepackage{parskip}
}{% else
\setlength{\parindent}{0pt}
\setlength{\parskip}{6pt plus 2pt minus 1pt}
}
\usepackage{hyperref}
\hypersetup{
            pdftitle={Statistical Learning Final Project Plots},
            pdfauthor={Daniel Alonso},
            pdfborder={0 0 0},
            breaklinks=true}
\urlstyle{same}  % don't use monospace font for urls
\usepackage[margin=1in]{geometry}
\usepackage{graphicx,grffile}
\makeatletter
\def\maxwidth{\ifdim\Gin@nat@width>\linewidth\linewidth\else\Gin@nat@width\fi}
\def\maxheight{\ifdim\Gin@nat@height>\textheight\textheight\else\Gin@nat@height\fi}
\makeatother
% Scale images if necessary, so that they will not overflow the page
% margins by default, and it is still possible to overwrite the defaults
% using explicit options in \includegraphics[width, height, ...]{}
\setkeys{Gin}{width=\maxwidth,height=\maxheight,keepaspectratio}
\setlength{\emergencystretch}{3em}  % prevent overfull lines
\providecommand{\tightlist}{%
  \setlength{\itemsep}{0pt}\setlength{\parskip}{0pt}}
\setcounter{secnumdepth}{0}
% Redefines (sub)paragraphs to behave more like sections
\ifx\paragraph\undefined\else
\let\oldparagraph\paragraph
\renewcommand{\paragraph}[1]{\oldparagraph{#1}\mbox{}}
\fi
\ifx\subparagraph\undefined\else
\let\oldsubparagraph\subparagraph
\renewcommand{\subparagraph}[1]{\oldsubparagraph{#1}\mbox{}}
\fi

% set default figure placement to htbp
\makeatletter
\def\fps@figure{htbp}
\makeatother

\usepackage{xcolor}

\title{Statistical Learning Final Project Plots}
\author{Daniel Alonso}
\date{December 19th, 2020}

\begin{document}
\maketitle

{
\setcounter{tocdepth}{4}
\tableofcontents
}
\newpage

\hypertarget{dataset-of-choice}{%
\section{Dataset of choice}\label{dataset-of-choice}}

For this project I decided to pick a custom-built dataset obtained from
the \href{https://databank.worldbank.org/home.aspx}{World bank
Databank}, specifically the
\href{https://databank.worldbank.org/source/world-development-indicators}{World
Development Indicators database}. This is the ``primary World Bank
collection of development indicators'' as stated on the database
description. It has lots of economic, education, energy use, and
population specific metrics.

I find demographic data fascinating, and I think this dataset will be
quite good for predicting country development measures along with
providing quite interesting and relevant information.

\hypertarget{variables}{%
\subsection{Variables}\label{variables}}

\large

NOTE: \normalsize

\begin{itemize}
\tightlist
\item
  \textbf{\textcolor{blue}{blue}} = used for training/predicting
\item
  \textbf{\textcolor{red}{red}} = target variable
\item
  \textbf{\textcolor{green}{green}} = ID variables
\item
  \textbf{\textcolor{violet}{purple}} = variables excluded as either
  they were \emph{components} of HDI or they were 100\% correlated to
  another variable (like GNI/GDP, which are both 100\% correlated and
  GNI is a component of HDI)
\end{itemize}

Variables in the original dataset as constructed using the World Bank
Databank tool (variables were renamed):

\begin{itemize}
\tightlist
\item
  \textcolor{green}{\textbf{year}}: year the data was obtained in
\item
  \textcolor{green}{\textbf{year\_code}}: code for the year as the world
  bank databank sets it
\item
  \textcolor{green}{\textbf{country\_name}}: name of the country
\item
  \textcolor{green}{\textbf{country\_code}}: alpha-3 ISO 3166 code for
  the country
\item
  \textcolor{blue}{\textbf{foreign\_inv\_inflows}}: Foreign direct
  investment, net inflows (BoP, current US\$)
\item
  \textcolor{blue}{\textbf{exports\_perc\_gdp}}: Exports of goods and
  services (as a \% of GDP)
\item
  \textcolor{blue}{\textbf{inflation\_perc}}: Inflation, consumer prices
  (annual \%)
\item
  \textcolor{blue}{\textbf{education\_years}}: Compulsory education,
  duration (years)
\item
  \textcolor{blue}{\textbf{education\_perc\_gdp}}: Government
  expenditure on education, total (as a \% of GDP)
\item
  \textcolor{blue}{\textbf{gds\_perc\_gdp}}: Gross domestic savings (as
  a \% of GDP)
\item
  \textcolor{blue}{\textbf{gross\_savings\_perc\_gdp}}: Gross savings
  (as a \% of GDP)
\item
  \textcolor{blue}{\textbf{int\_tourism\_arrivals}}: International
  tourism, number of arrivals
\item
  \textcolor{blue}{\textbf{int\_tourism\_receipts}}: International
  tourism, receipts (in current US\$)
\item
  \textcolor{blue}{\textbf{perc\_internet\_users}}: Individuals using
  the Internet (as a \% of population)
\item
  \textcolor{blue}{\textbf{access\_to\_electricity}}: Access to
  electricity (\% of population)
\item
  \textcolor{blue}{\textbf{agricultural\_land}}: Agricultural land (\%
  of land area)
\item
  \textcolor{blue}{\textbf{birth\_rate}}: Birth rate, crude (per 1,000
  people)
\item
  \textcolor{blue}{\textbf{gne}}: Gross national expenditure (\% of GDP)
\item
  \textcolor{blue}{\textbf{mobile\_subscriptions}}: Mobile cellular
  subscriptions (per 100 people)
\item
  \textcolor{blue}{\textbf{infant\_mort\_rate}}: Mortality rate, infant
  (per 1,000 live births)
\item
  \textcolor{blue}{\textbf{sex\_ratio}}: Sex ratio at birth (male births
  per female births)
\item
  \textcolor{blue}{\textbf{greenhouse\_gas\_em}}: Total greenhouse gas
  emissions (kt of CO2 equivalent)
\item
  \textcolor{blue}{\textbf{urban\_pop\_perc}}: Urban population (\% of
  total population)
\item
  \textcolor{red}{\textbf{hdi}}: human development index
\item
  \textcolor{red}{\textbf{hdi\_cat}}: Human development index as a
  category
\item
  \textcolor{violet}{\textbf{life\_exp}}: Life expectancy at birth,
  total (years)
\item
  \textcolor{violet}{\textbf{gdp}}: GDP (current US\$)
\item
  \textcolor{violet}{\textbf{gni}}: GNI (current US\$)
\item
  \textcolor{violet}{\textbf{fertility\_rate}}: Fertility rate, total
  (births per woman)
\end{itemize}

\newpage

\hypertarget{the-target-variable}{%
\subsubsection{The target variable}\label{the-target-variable}}

As all these variables could perhaps tell us how developed a country is,
we used a constructed categorized Human development index variable in
order to classify the countries using the above variables (unless stated
otherwise by their colour) as training variables.

The criteria for constructing the categorical variable \emph{hdi\_cat}
was the following:

\begin{itemize}
    \item Very high: HDI above 0.8
    \item High: HDI between 0.7 and 0.799
    \item Medium: HDI between 0.55 and 0.699
    \item Low: HDI under 0.55
  \end{itemize}

This categorization is emulated from Wikipedia's construction and uses
the same ranges as used in every Wikipedia article referencing HDI.

\hypertarget{data-preprocessing}{%
\section{Data preprocessing}\label{data-preprocessing}}

The data cleanup and preliminary feature engineering was done in Python
and the imputation was then done in R within the same Jupyter Notebook
(called \emph{preprocessing.ipynb}).

\hypertarget{methodology-and-steps}{%
\subsection{Methodology and steps}\label{methodology-and-steps}}

\begin{enumerate}
  \item Prior to importing there was a search within the .csv file with the following regex: \begin{verbatim} \s\[[\w\S]{1,}\] \end{verbatim} as a find and replace and removing every instance of it. This regex matches a metadata tag that the world bank uses in their dataset. Following this step, the data was imported.

  \item Metadata at the end of the dataset was removed, the dataset was filtered excluding these region codes (the codes were excluded as they were country aggregates, and we're only interested in the countries themselves).

  \item The year column was converted to integer and the '..' values (which represent NAs in the World bank databank) were replaced for numpy NaNs and then the values were sorted by year and country name.

  \item Data missing in later years (2020, 2019) was backfilled with data from previous years, as it still fits our modelling purposes. The data that goes the furthest back is from ~18 years prior, so 2002.

  \item Columns with more than 45 NA values were removed as this represents roughly 25% of the countries in question.

  \item The life expectancy, GDP, GNI and fertility rate columns were removed as these are either components of HDI or (in the case of fertility rate) are 100% correlated with other elements in our dataset (crude birth rate in this particular case).

  \item We scrape Wikipedia in order to obtain an updated metric on HDI estimates for each country. We also scrape it to obtain the alpha-3 3166 codes for each country. The data was obtained with the purpose of making it easier to join with the dataset obtained from the World Bank, as joining by country names is not reliable enough (misses ~20 countries which the alpha-3 codes do not).

  \item We add HDI to the main dataframe

  \item Numerical columns were converted into floats and column names were simplified for easier future manipulation.

  \item The categorical target variable is constructed as explained previously in this step.

  \item The data is then exported as a csv, along with a json file that shows how the columns were renamed

  \item The data is then imported in an R cell and a MICE imputation is performed using the 'cart' method with m = 5 in order to impute the very few missing values remaining.
\end{enumerate}

\end{document}
